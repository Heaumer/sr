\documentclass[a4paper]{article}

\usepackage[francais]{babel}
\usepackage[utf8]{inputenc}
\usepackage[OT1]{fontenc}
\usepackage{amsmath}
\usepackage{amssymb}
\usepackage{graphicx}
\usepackage{url}
\usepackage{subfigure}

\graphicspath{ {./imgs/} }

% shorten margin
\usepackage[]{fullpage}

\title{Rapport des TDs du cours de Sécurité Réseau par Bruno Martin}
\author{Sophie Valentin, Mathieu Bivert}

\makeatletter
\def\thickhrulefill{\leavevmode \leaders \hrule height 1pt\hfill \kern \z@}
\def\maketitle{
	\null
	\thispagestyle{empty}
	\vskip 1cm
	\begin{center}
		\normalfont\large\huge\@author
	\end{center}
	\vfil
	\vfil
	\vfil
	\vfil
	\vfil
	\vfil
	\vfil
	\vfil
	\vfil
	\hrule height 2pt
	\par
	\begin{center}
				\huge \strut \@title \par
				\@date
	\end{center}
	\hrule height 2pt
	\par
	\vfil
	\vfil
	\vfil
	\vfil
	\vfil
	\vfil
	\vfil
	\vfil
	\vfil
	\vfil
	\vfil
	\vfil
	\vfil
	\vfil
	\vfil
	\vfil
	\vfil
	\vfil
	\vfil
	\vfil
	\vfil
	\vfil
	\vfil
	\vfil
	\vfil
	\null
	\begin{figure}[!ht]
		\centering
		\includegraphics[scale=.5]{polytech.png}
	\end{figure}
	\vfil
	\cleardoublepage
}
\makeatother

\begin{document}
\maketitle

\newpage
\tableofcontents

\newpage

\section{Topologie}
\section{Configuration de la passerelle}
Une passerelle (gateway) est un homme du milieu reliant
deux réseaux distincts. Dans le cas présent, la machine \textit{passerelle}
doit faire communiquer les deux réseaux SLAN et LAN Travaux Pratiques.

La passerelle doit être capable de router du traffic:
\begin{verbatim}
  (passerelle)# echo 1 > /proc/sys/net/ipv4/ip_forward
\end{verbatim}

L'IP Masquerade (Network Address Translation) doit être
activée. Cette fonctionnalité modifie les entêtes IPs du traffic
passant par \textit{passerelle} afin de rendre invisibles, au niveau IP,
les machines de LAN Travaux Pratiques depuis l'extérieur.

\begin{verbatim}
  (passerelle)# iptables -t nat -A POSTROUTING -o eth0 -s 192.168.2.0/24 -j MASQUERADE
\end{verbatim}

Enfin, syslogd doit être activé afin de logger les activités d'iptables

\begin{verbatim}
  (passerelle)# apt-get install inetutils-syslogd
  (passerelle)# edit /etc/syslog.conf # logs dans /var/log/kernel.log
  (passerelle)# services syslog
\end{verbatim}

\subsection{Tests}
On choisit un client, par exemple \textit{client-bsd}, on lui
retire l'interface réseau connectée à SLAN, et on s'assure que
la machine est bien connectée sur LAN Travaux Pratiques, et
qu'elle peut communiquer avec la passerelle. Enfin, on s'assure
qu'il est possible de contacter le serveur et les Internets.

\begin{verbatim}
  (client-bsd)# ifconfig em1 down # ou em0
  (client-bsd)# ifconfig em0      # ou em1
  (client-bsd)# ping passerelle.cs.sr
  (client-bsd)# netstat -r 
  (client-bsd)# ping -c3 google.fr
\end{verbatim}

On vérifie les logs sur la passerelle:
\begin{verbatim}
  (passerelle)# tail -f /var/log/kernel.log
\end{verbatim}

serveur telnet. mitm depuis backtrack. ssh. nmap

essai ssh avec mitm? :-)

\section{Configuration du serveur web}
\subsection{Tests}

\section{Configuration serveur (smtp+imaps) et client (gpg) email}
routage email, topologie
\subsection{Tests}

\section{Configuration VPN}
\subsection{Tests}

\section{OpenVAS \& Metasploit}

\end{document}
